\documentclass[a4paper,14pt]{extarticle}

\def\labauthors{Сарафанов Ф.Г., Платонова М.В.}
\def\labgroup{440}
\def\labnumber{1}
\def\labstartdate{5 ноября}
\def\labtheme{Измерение статических характеристик \\[0.2em] полевого транзистора}
\def\shortlabtheme{Полевой транзистор}

\input{text/defpart}
\begin{document}
\input{text/titlepage}

\tableofcontents
\newpage



\addcontentsline{toc}{section}{Введение}
\section*{Введение}

В настоящей работе определяется  ы
с помощью ы-метода.  Как показано в \cite[стр. ы]{met}

\paragraph{Установка.} 
Все измерения базируются на получении  ы.
При измерениях снимаются показания ы.

\paragraph{Ы-метод.} 
Для реализации ы-метода используется ы.

Все измерения проводятся таким образом, чтобы ы.

\newpage

\section{Расчет ы}


\section{Теоретическое ы}


\newpage
\section{Экспериментальное ы}

\begin{figure}[h!]
    \centering
    \includegraphics[scale=1.5]{fig/e2_from_x}
    \caption{Зависимость ы от  ы}
    \label{fig:exp:1}
\end{figure}




\addcontentsline{toc}{section}{Заключение}
\section*{Заключение}
В настоящей работе мы измерили ы.
Ы при ряде приближений ы.
Расхождение с практическими результатами объясняется ы.



\begin{thebibliography}{}
  \bibitem{orlov} Орлов И.\,Я., Односевцев В.\,А. и др. Основы радиоэлектроники: учебное пособие. -- Нижний Новгород: Нижегородский государственный университет им. Н.И. Лобачевского, 2011. -- 169 с.
  
  \bibitem{met} Битюрин\,\,Ю.\,А. и др. Измерение статических характеристик полевого транзистора. Н.Новгород: ННГУ, 2003. -- 30 с.
  
  % \bibitem{lit3} Ландау Л.Д., Лифшиц Е.М. Любой том. М.: Физматлит, 2003.
\end{thebibliography}

\end{document}